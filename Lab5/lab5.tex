
% This LaTeX was auto-generated from MATLAB code.
% To make changes, update the MATLAB code and republish this document.

\documentclass{article}
\usepackage{graphicx}
\usepackage{color}
\usepackage{listings}
\usepackage{hyperref}
\usepackage{multicol, caption}
\usepackage{amsmath}
\usepackage{textcomp}

\newenvironment{Figure}
	{\par\medskip\noindent\minipage{\linewidth}}
	{\endminipage\par\medskip}

% Define new length
\newlength{\figureWidth}
% Set the new length to a specific value
\setlength{\figureWidth}{3in}

\newcommand{\makeFigure}[3]{
	\begin{Figure}
		\centering
		\includegraphics [width=\figureWidth]{html/{#1}}
		\captionof{figure}{#2}
		\label{fig:#3}
	\end{Figure}
}

\setlength{\columnsep}{2cm}
\usepackage[a4paper, total={7in, 8.5in}]{geometry}

\setlength{\topmargin}{-0.6in}
\setlength{\textwidth}{7in}
\setlength{\textheight}{9.7in}

\definecolor{mygreen}{RGB}{28,172,0} % color values Red, Green, Blue
\definecolor{mylilas}{RGB}{170,55,241}

\lstset{language=Matlab,%
	%basicstyle=\color{red},
	basicstyle=\footnotesize\texttt{},
	upquote=true,%
	breaklines=true,%
	tabsize=2,
	morekeywords={matlab2tikz},
	keywordstyle=\color{blue},%
	morekeywords=[2]{1}, keywordstyle=[2]{\color{black}},
	identifierstyle=\color{black},%
	stringstyle=\color{mylilas},
	commentstyle=\color{mygreen},%
	showstringspaces=false,%without this there will be a symbol in the places where there is a space
	numbers=left,%
	numberstyle={\tiny \color{black}},% size of the numbers
	numbersep=9pt, % this defines how far the numbers are from the text
	emph=[1]{for,end,break},emphstyle=[1]\color{red}, %some words to emphasise
	%emph=[2]{word1,word2}, emphstyle=[2]{style},
}

\definecolor{lightgray}{gray}{0.5}
\setlength{\parindent}{0pt}


\begin{document}
	
	\title{Machine Learning Lab Five COMP3206}
	\author{Alois Klink \and \href{mailto:ak9g14@soton.ac.uk}{ak9g14@soton.ac.uk} \and 2704 7555}
	\maketitle
	
\begin{multicols}{2}[
	\begin{par}
		This Lab covers Radial Basis Function Models. An explanation of what they are can be found in Section \ref{sec:radBasis}. The lab notes can be found at: 
		\url{https://secure.ecs.soton.ac.uk/notes/comp3206/ml_lab_five2016.pdf}
	\end{par}
	]
\section*{The Radial Basis Function Model}
\label{sec:radBasis}
\begin{par}
	A radial basis function model uses the sum of many weighted radial
	basis functions to approximate a function, where each radial basis function is some
	fucntion of distance between the function input and a point from the model.
\end{par}

\lstinputlisting[language=Matlab]{radialBasis.m}


\section*{Results for K = Ntr/10}

\begin{lstlisting}
K = round(size(Xtr,1)/10);
ytrpredict = radialBasis(Xtr, ytr, basisFunction, K, Xtr);
ytstpredict = radialBasis(Xtr, ytr, basisFunction, K, Xtst);
plot(-2:2,-2:2, 'DisplayName', 'Zero-Error Line');
hold on; axis([-2 3.1 -2.5 2])
plot(ytr, ytrpredict, 'r.', 'LineWidth', 2, 'DisplayName', 'Training Data');
plot(ytst, ytstpredict, 'bx', 'LineWidth', 2, 'DisplayName', 'Test Data'), grid on
title('RBF Prediction on Data', 'FontSize', 16);
xlabel('Target', 'FontSize', 14);
ylabel('Prediction', 'FontSize', 14);
legend('Location', 'Southeast');
hold off;
\end{lstlisting}

\makeFigure{lab5_01.eps}{Shows the difference between the real data and the predicted data from the RBF model}{1}


\section*{RMS Error for a Range of Ks}

\begin{par}
	The following code plots the root-mean-square (RMS) error on the RBF model's output, when different K's are used. As can be seen, it appears that oddly enough, the test data does pretty well, until K reaches around 10, and the model overfits towards the training data, becoming too optimized towards the training data for real purposes. 
\end{par}  
\begin{lstlisting}
rms = @(input) sqrt(mean(input.^2));
for K = 1:110
ytrpredict = radialBasis(Xtr, ytr, basisFunction, K, Xtr);
trError(K) = rms(ytrpredict-ytr);
ytstpredict = radialBasis(Xtr, ytr, basisFunction, K, Xtst);
tstError(K) = rms(ytstpredict-ytst);
end
K = 1:110;
plot(K, trError, 'LineWidth', 2, 'DisplayName', 'Training RMS Error'); hold on;
plot(K, tstError, 'LineWidth', 2, 'DisplayName', 'Test RMS Error'), grid on
title('RMS Error on Data with differing K', 'FontSize', 16);
xlabel('K', 'FontSize', 14);
ylabel('RMS Error', 'FontSize', 14);
legend('Location', 'Northeast');
hold off;
\end{lstlisting}

\makeFigure{lab5_02.eps}{Shows how RMS Error differs when K differs.}{2}


\section*{Comparing Linear Least Square Regression to RBF on Housing Data}

\begin{par}
	RBF seems to be worse than Linear Least Square Regression, at least with $ K = \frac{N_tr}{10} $.
\end{par}

\makeFigure{lab5_03.eps}{Shows the RMS Error on UCI's Housing data. The RMS Error is obtained from 20 random partitionings.}{3}

\section*{Comparing Linear Least Square Regression to RBF on Power Plant}

\makeFigure{lab5_04.eps}{Shows the RMS Error on UCI's Combined Cycle Power Plant data. The RMS Error is obtained from 20 random partitionings.}{4}

\end{multicols}

\end{document}
    
exit