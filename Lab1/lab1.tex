
% This LaTeX was auto-generated from MATLAB code.
% To make changes, update the MATLAB code and republish this document.

\documentclass{article}
\usepackage{graphicx}
\usepackage{color}
\usepackage{listings}
\usepackage{hyperref}
\usepackage{multicol}
\setlength{\columnsep}{2cm}
\usepackage[a4paper, total={7in, 8.5in}]{geometry}

\setlength{\topmargin}{-0.5in}
\setlength{\textwidth}{7in}
\setlength{\textheight}{9.5in}

% Define new length
\newlength{\figureWidth}
% Set the new length to a specific value
\setlength{\figureWidth}{2.6in}

\definecolor{mygreen}{RGB}{28,172,0} % color values Red, Green, Blue
\definecolor{mylilas}{RGB}{170,55,241}

\lstset{language=Matlab,%
	%basicstyle=\color{red},
	basicstyle=\footnotesize\texttt{},
	breaklines=true,%
	morekeywords={matlab2tikz},
	keywordstyle=\color{blue},%
	morekeywords=[2]{1}, keywordstyle=[2]{\color{black}},
	identifierstyle=\color{black},%
	stringstyle=\color{mylilas},
	commentstyle=\color{mygreen},%
	showstringspaces=false,%without this there will be a symbol in the places where there is a space
	numbers=left,%
	numberstyle={\tiny \color{black}},% size of the numbers
	numbersep=9pt, % this defines how far the numbers are from the text
	emph=[1]{for,end,break},emphstyle=[1]\color{red}, %some words to emphasise
	%emph=[2]{word1,word2}, emphstyle=[2]{style},    
}

\sloppy
\definecolor{lightgray}{gray}{0.5}
\setlength{\parindent}{0pt}


\begin{document}

    
    
\title{Machine Learning Lab One COMP3206}
\author{Alois Klink \and \href{mailto:ak9g14@soton.ac.uk}{ak9g14@soton.ac.uk} }
\maketitle

\begin{multicols}{2}[
\begin{par}
This Lab covers generating 2-D normal (Gaussian) distributions.
The lab notes can be found here:
\url{https://secure.ecs.soton.ac.uk/notes/comp6229/ml_lab_one2016.pdf}
\end{par}
]
\subsection*{Step 2}

\begin{par}
The following shows a histogram of N=1000 uniform random numbers. The histogram has 10 categories with equal spread. As can be seen, because of the random nature of the data, the histrogram is not prefectly flat, however, it is pretty accurate.
\end{par} \vspace{1em}
\begin{lstlisting}
N = 1000; x = rand(N,1);
bar(hist(x,10)); %one-liner to save space since two $$/Latex$$ pages is no space
xlabel('Number of Category'); 
ylabel('Size of Category'); 
title('Uniform Random Numbers Histogram');
\end{lstlisting}

\includegraphics [width=\figureWidth]{labone_01.eps}
\begin{par}
The following shows a histogram of N=1000 normal distrubuted random numbers. (Also known as a Gaussian distribution). The histogram has 10 categories with equal spread. It follows the sterotypical normal distribution `bell' curve.
\end{par} \vspace{1em}
\begin{lstlisting}
x = randn(N, 1);
bar(hist(x,10));
xlabel('Number of Category'); 
ylabel('Size of Category'); 
title('Normal Distributed Numbers Histogram');
\end{lstlisting}

\includegraphics [width=\figureWidth]{labone_02.eps}
\begin{par}
The following shows a histogram of the sum of 12 uniform distributed random numbers minus the sum of 12 other uniform distributed random numbers.
\end{par} \vspace{1em}
\begin{par}
This shows a Gaussian distribution, even though only uniform distributed values were input. This shows the \href{https://en.wikipedia.org/wiki/Central_limit_theorem}{Central Limit Theorem} in action.
\end{par} \vspace{1em}
\begin{lstlisting}
x1 = zeros(N,1);
for n=1:N
	x1(n,1) = sum(rand(12,1))-sum(rand(12,1));
end
hist(x1,40);
xlabel('Category Value'); ylabel('Size of Category'); title('Normal Distributed Numbers Histogram');
\end{lstlisting}

\includegraphics [width=\figureWidth]{labone_03.eps}


\subsection*{Step 3 and 4}

\begin{par}
The following factorizes \textbf{C} into \textbf{A}, which can be confirmed by looking at \textbf{Ctest}.
\end{par} \vspace{1em}
\begin{lstlisting}
C = [2 1 ; 1 2]
A = chol(C); Ctest = transpose(A) * A
\end{lstlisting}

        \color{lightgray} \begin{lstlisting}
C =  	2     1
        1     2
     
Ctest = 2.0000    1.0000
        1.0000    2.0000

\end{lstlisting} \color{black}
    \begin{par}
The graph shows that \textbf{X} is isotropically Normally distributed, while \textbf{Y} is distributed roughly correlated to a line.
\end{par} \vspace{1em}
\begin{lstlisting}
X = randn(N,2);
Y = X * A;
plot(X(:,1),X(:,2),'c.', Y(:,1),Y(:,2),'mx');
xlabel('X_1 features'); ylabel('X_2 features'); title('Scatter Plot of X and Y');
\end{lstlisting}

\includegraphics [width=\figureWidth]{labone_04.eps}
\begin{par}
This graph shows \textbf{Y} projected onto a line \textbf{W} with direction $\theta$. It then plots the variance of this for each $\theta$. This makes a sine wave, whose maxima and minima correspond to \textbf{C}'s eigenvalues, and the turning points correspond to the location of \textbf{C}'s eigenvectors.
\end{par} \vspace{1em}
\begin{par}
Using the other covariance matrix, \textbf{C2}, the sine wave has a $\pi$ added phase.
\end{par} \vspace{1em}
\begin{lstlisting}
Nvals = 50;
plotArray = zeros(Nvals,1);
yp = zeros(N, 1, Nvals);
thetaRange = linspace(0,2*pi,Nvals);
for n=1:Nvals
    theta = thetaRange(n);
    W = [sin(theta); cos(theta)];
    yp(:,:, n) = Y * W;
    plotArray(n) = var( yp(:,:, n) );
end
plot(thetaRange, plotArray); ylim([0.8, 3.2]);
xlabel('Angle of Projection Line'); ylabel('Variance'); title('Variance of Y when Projected');
set(gca,'xtick',0:pi/2:2*pi); set(gca,'xticklabel',{'0','\pi/2','\pi','3 \pi/2','2 \pi'});
hold on;

%
% Using Y2, which is calculated from C2, this graph shows Y2 projected onto
% a line W with direction theta. It then
% plots the variance of this for each theta.
%
C2 = [2 -1; -1 2];
A2 = chol(C2);
Y2 = X * A2;
plotArray2 = zeros(Nvals,1);
yp2 = zeros(N, 1, Nvals);
for n=1:Nvals
    theta = thetaRange(n);
    W = [sin(theta); cos(theta)];
    yp2(:,:, n) = Y2 * W;
    plotArray2(n) = var( yp2(:,:, n) );
end
plot(thetaRange, plotArray2, '-');

% Find the Eigenvectors and Eigenvals and plot them
[eigVecs, eigVals] = eig(C);
eigVecAngs(1) = atan(eigVecs(1,1)/eigVecs(2,1)) + 2*pi;
eigVecAngs(2) = atan(eigVecs(1,2)/eigVecs(2,2));
plot([0 2*pi], [eigVals(1) eigVals(1)]); plot([0 2*pi], [eigVals(4) eigVals(4)]);
plot(eigVecAngs(1)*[1 1], [1 3]); plot(eigVecAngs(2)*[1 1], [1 3]);
hold off;
\end{lstlisting}

\includegraphics [width=\figureWidth]{labone_05.eps}

\end{multicols}



\end{document}
    
